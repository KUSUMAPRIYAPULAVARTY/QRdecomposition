\documentclass[journal,12pt,twocolumn]{IEEEtran}

\usepackage{setspace}
\usepackage{gensymb}

\singlespacing


\usepackage[cmex10]{amsmath}

\usepackage{amsthm}

\usepackage{mathrsfs}
\usepackage{txfonts}
\usepackage{stfloats}
\usepackage{bm}
\usepackage{cite}
\usepackage{cases}
\usepackage{subfig}

\usepackage{longtable}
\usepackage{multirow}

\usepackage{enumitem}
\usepackage{mathtools}
\usepackage{steinmetz}
\usepackage{tikz}
\usepackage{circuitikz}
\usepackage{verbatim}
\usepackage{tfrupee}
\usepackage[breaklinks=true]{hyperref}
\usepackage{graphicx}
\usepackage{tkz-euclide}

\usetikzlibrary{calc,math}
\usepackage{listings}
    \usepackage{color}                                            %%
    \usepackage{array}                                            %%
    \usepackage{longtable}                                        %%
    \usepackage{calc}                                             %%
    \usepackage{multirow}                                         %%
    \usepackage{hhline}                                           %%
    \usepackage{ifthen}                                           %%
    \usepackage{lscape}     
\usepackage{multicol}
\usepackage{chngcntr}

\DeclareMathOperator*{\Res}{Res}

\renewcommand\thesection{\arabic{section}}
\renewcommand\thesubsection{\thesection.\arabic{subsection}}
\renewcommand\thesubsubsection{\thesubsection.\arabic{subsubsection}}

\renewcommand\thesectiondis{\arabic{section}}
\renewcommand\thesubsectiondis{\thesectiondis.\arabic{subsection}}
\renewcommand\thesubsubsectiondis{\thesubsectiondis.\arabic{subsubsection}}


\hyphenation{op-tical net-works semi-conduc-tor}
\def\inputGnumericTable{}                                 %%

\lstset{
%language=C,
frame=single, 
breaklines=true,
columns=fullflexible
}
\begin{document}


\newtheorem{theorem}{Theorem}[section]
\newtheorem{problem}{Problem}
\newtheorem{proposition}{Proposition}[section]
\newtheorem{lemma}{Lemma}[section]
\newtheorem{corollary}[theorem]{Corollary}
\newtheorem{example}{Example}[section]
\newtheorem{definition}[problem]{Definition}

\newcommand{\BEQA}{\begin{eqnarray}}
\newcommand{\EEQA}{\end{eqnarray}}
\newcommand{\define}{\stackrel{\triangle}{=}}
\bibliographystyle{IEEEtran}

\providecommand{\mbf}{\mathbf}
\providecommand{\pr}[1]{\ensuremath{\Pr\left(#1\right)}}
\providecommand{\qfunc}[1]{\ensuremath{Q\left(#1\right)}}
\providecommand{\sbrak}[1]{\ensuremath{{}\left[#1\right]}}
\providecommand{\lsbrak}[1]{\ensuremath{{}\left[#1\right.}}
\providecommand{\rsbrak}[1]{\ensuremath{{}\left.#1\right]}}
\providecommand{\brak}[1]{\ensuremath{\left(#1\right)}}
\providecommand{\lbrak}[1]{\ensuremath{\left(#1\right.}}
\providecommand{\rbrak}[1]{\ensuremath{\left.#1\right)}}
\providecommand{\cbrak}[1]{\ensuremath{\left\{#1\right\}}}
\providecommand{\lcbrak}[1]{\ensuremath{\left\{#1\right.}}
\providecommand{\rcbrak}[1]{\ensuremath{\left.#1\right\}}}
\theoremstyle{remark}
\newtheorem{rem}{Remark}
\newcommand{\sgn}{\mathop{\mathrm{sgn}}}
\providecommand{\abs}[1]{\left\vert#1\right\vert}
\providecommand{\res}[1]{\Res\displaylimits_{#1}} 
\providecommand{\norm}[1]{\left\lVert#1\right\rVert}
%\providecommand{\norm}[1]{\lVert#1\rVert}
\providecommand{\mtx}[1]{\mathbf{#1}}
\providecommand{\mean}[1]{E\left[ #1 \right]}
\providecommand{\fourier}{\overset{\mathcal{F}}{ \rightleftharpoons}}
%\providecommand{\hilbert}{\overset{\mathcal{H}}{ \rightleftharpoons}}
\providecommand{\system}{\overset{\mathcal{H}}{ \longleftrightarrow}}
	%\newcommand{\solution}[2]{\textbf{Solution:}{#1}}
\newcommand{\solution}{\noindent \textbf{Solution: }}
\newcommand{\cosec}{\,\text{cosec}\,}
\providecommand{\dec}[2]{\ensuremath{\overset{#1}{\underset{#2}{\gtrless}}}}
\newcommand{\myvec}[1]{\ensuremath{\begin{pmatrix}#1\end{pmatrix}}}
\newcommand{\mydet}[1]{\ensuremath{\begin{vmatrix}#1\end{vmatrix}}}

\numberwithin{equation}{subsection}

\makeatletter
\@addtoreset{figure}{problem}
\makeatother
\let\StandardTheFigure\thefigure
\let\vec\mathbf

\renewcommand{\thefigure}{\theproblem}

\def\putbox#1#2#3{\makebox[0in][l]{\makebox[#1][l]{}\raisebox{\baselineskip}[0in][0in]{\raisebox{#2}[0in][0in]{#3}}}}
     \def\rightbox#1{\makebox[0in][r]{#1}}
     \def\centbox#1{\makebox[0in]{#1}}
     \def\topbox#1{\raisebox{-\baselineskip}[0in][0in]{#1}}
     \def\midbox#1{\raisebox{-0.5\baselineskip}[0in][0in]{#1}}
\vspace{3cm}
\title{QR decomposition}
\author{KUSUMA PRIYA\\EE20MTECH11007}

\maketitle
\newpage

\bigskip
\renewcommand{\thefigure}{\theenumi}
\renewcommand{\thetable}{\theenumi}
Download latex codes from 

\begin{lstlisting}
https://github.com/KUSUMAPRIYAPULAVARTY/QRdecomposition
\end{lstlisting}
%
 
 \section{QUESTION}
Perform the QR decomposition of matrix \vec{A}
\begin{align}
 \vec{A}=\myvec{1&2\\3&1}
\end{align}

%
\section {Explanation}
If $\alpha$ and $\beta$ are the columns of a (2$\times$2) matrix $\vec{A}$,\\
then \vec{A} can be decomposed as 
\begin{align}
  \vec{A}=\vec{Q}\vec{R}\label{1}\\
  \text{where, } \vec{U}=\myvec{\vec{u_1}&\vec{u_2}},\\
  \text{uppertriangular matrix }\vec{R}=\myvec{k_1&r_1\\0&k_2}\\
  k_1=\norm{\alpha},\vec{u_1}=\frac{\alpha}{k_1}\label{2}\\
 r_1=\frac{\vec{u_1}^T\beta}{\norm{\vec{u_1}}^2}\label{3} \\
 \vec{u_2}=\frac{\beta-r_1\vec{u_1}}{\norm{\beta-r_1\vec{u_1}}},k_2=\vec{u_2}^T\beta\label{4}
\end{align}

\section{Solution}
\begin{align}
\alpha=\myvec{1\\3},\beta=\myvec{2\\1}\\
\text{From, \eqref{2}, } k_1=\norm{\alpha}=\sqrt{10}\\
\text{and } \vec{u_1}=\frac{1}{\sqrt{10}}\myvec{1\\3}\\
\text{From \eqref{3}, }r_1=\frac{1}{\sqrt{10}}\myvec{1&3}\myvec{2\\1}=\frac{5}{\sqrt{10}}
\end{align}
\begin{align}
\beta-r_1\vec{u_1}=\myvec{2\\1}-\frac{5}{\sqrt{10}}\frac{1}{\sqrt{10}}\myvec{1\\3}\\
=\myvec{\frac{3}{2}\\\frac{-1}{2}}\\
\text{From \eqref{4}, }\vec{u_2}=\frac{\myvec{\frac{3}{2}\\\frac{-1}{2}}}{\sqrt{\frac{9}{4}+\frac{1}{4}}}\\
\implies \vec{u_2}=\myvec{\frac{3}{\sqrt{10}}\\\frac{-1}{\sqrt{10}}},\\
k_2=\myvec{\frac{3}{\sqrt{10}}&\frac{-1}{\sqrt{10}}}\myvec{2\\1}=\frac{5}{\sqrt{10}}
\end{align}
Note that, 
\begin{align}
    \vec{Q}^T\vec{Q}=\myvec{\frac{1}{\sqrt{10}}&\frac{3}{\sqrt{10}}\\\frac{3}{\sqrt{10}}&\frac{-1}{\sqrt{10}}}\myvec{\frac{1}{\sqrt{10}}&\frac{3}{\sqrt{10}}\\\frac{3}{\sqrt{10}}&\frac{-1}{\sqrt{10}}}=\myvec{1&0\\0&1}=\vec{I}
\end{align}
The matrix \vec{A} can now be rewritten using \eqref{1} as
\begin{align}
  \myvec{1&2\\3&1}=\myvec{\frac{1}{\sqrt{10}}&\frac{3}{\sqrt{10}}\\\frac{3}{\sqrt{10}}&\frac{-1}{\sqrt{10}}}\myvec{\sqrt{10}&\frac{5}{\sqrt{10}}\\0&\frac{5}{\sqrt{10}}}
\end{align}
\end{document}


